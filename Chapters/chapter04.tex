\section{Operationally Accessible Entanglement Entropy}

\subsection{The \ren Entanglement Entropy}

The amount of entanglement that exists between some partition $A$ and its compliment $\bar{A}$ of a quantum many-body system in pure state $\ket{\Psi}$ can be quantified via the R\'{e}nyi entanglement entropy which depends on an index $\alpha$ :
%
\begin{equation}
S_{\alpha} (\rho_A) = \frac{1}{1-\alpha}\ln \Tr\, \rho_A^{\alpha}
\label{eq:S_alpha}
\end{equation}
%
where $\rho_{A}$ is the reduced density matrix of partition $A$ obtained by
tracing out all degrees of freedom in $\bar{A}$ from the full density matrix:
%
\begin{equation}
\rho_{A} = \Tr_{\bar{A}}\, \rho = \Tr_{\bar{A}} \ket{\Psi}\bra{\Psi}\,.
\end{equation}
%
The \ren entropy is a monotonically decreasing function of $\alpha$ for $\alpha
> 1$ and is bounded from above by the von Neumann entropy, $S_1(\rho_A) = -\Tr \rho_A \ln \rho_A$.

For a quantum many-body system subject to physical laws conserving some quantity (particle number, charge, spin, etc.), the set of local operations on the state $\ket{\Psi}$ are limited to those that don't violate the corresponding global superselection rule.  For the remainder of this paper, we will focus on our discussion on the case of fixed total $N$ and thus we are restricted to only those operators which locally preserve the particle number in $A$.  The effect this has on the amount of entanglement that can be transferred to a qubit register is apparent from the simple example (adapted from Ref.~\onlinecite{Wiseman:2003vn} of one particle confined to two spatial modes $A$ and $\bar{A}$ corresponding to site occupations.  Then, for the state $\ket{\Psi} = \left(\ket{1}_A \otimes \ket{0}_{\bar{A}} + \ket{0}_A \otimes \ket{1}_{\bar{A}} \right)/\sqrt{2}$, Eq.~\eqref{eq:S_alpha} gives that $S_1 = \ln 2$. However, this entanglement cannot be transferred to a register prepared in initial state $\ket{0}_R$ via a $\texttt{SWAP}$ gate:
\begin{align*}
    & \texttt{SWAP} \ket{0}_R\otimes\left(\ket{1}_A \otimes
    \ket{0}_{\bar{A}} + \ket{0}_A \otimes \ket{1}_{\bar{A}} \right)/\sqrt{2} \\
    &= \frac{1}{\sqrt{2}}\left( \ket{0}_R \otimes \ket{0}_A \otimes
        \ket{1}_{\bar{A}} + \ket{1}_R \otimes \ket{0}_A \otimes
    \ket{0}_{\bar{A}} \right)
    % &= \frac{1}{\sqrt{2}}\ket{0}_R \otimes \ket{0}_A \otimes
    %     \ket{1}_{\bar{A}}
\end{align*}
where the last term is not physically allowed due to the restriction that the number of particles in the system is fixed to be 1. The post-swap result remains in a product state and the amount of transferable entanglement is identically zero.

\subsection{von Neumann Accessible Entanglement: $\alpha = 1$}
% ---------------------------------------------------------------------------

Thus, Eq.~\eqref{eq:S_alpha}, which includes the effects of non-local number fluctuations between $A$ and $\bar{A}$, overcounts the amount of entanglement that can be accessed from the system.  To quantify the physical reduction, Wiseman and Vaccaro \cite{Wiseman:2003jx} suggested that for the case of $\alpha = 1$ a more appropriate measure should weight contributions to the entanglement coming from each superselection sector corresponding to the number of particles $n$ in $A$:
%
\begin{equation}
    S_1^{\rm{acc}}(\rho_A)=\sum_{n=0}^{N} P_n S_1(\rho_{A_n})
\label{eq:S1acc}.
\end{equation}
%
Here $\rho_{A_{n}}$ is defined to be the reduced density matrix of $A$, projected onto the subspace of fixed local particle number $n$ 
%
\begin{equation}
    \rho_{A_n} = \frac{1}{P_n}{\mathcal{P}}_{A_n} \rho_{A} {\mathcal{P}}_{A_n}
\label{eq:rhoAn}
\end{equation}
%
accomplished via a projection operator ${\mathcal{P}}_{A_n}$ where
${\mathcal{P}}_{A_n} \ket{\Psi} = \ket{n}_A\otimes\ket{N-n}_{\bar{A}}$.  
$P_n$ is the probability of measuring $n$ particles in $A$:
%
\begin{equation}
    P_n = \mathrm{Tr}\, {\mathcal{P}}_{A_n} \rho_A{\mathcal{P}}_{A_n}
    = \expval{{\mathcal{P}}_{A_n}}{\Psi}.
    \label{eq:Pn}
\end{equation}
%
As the projection constitutes a local operation which can only decrease
entanglement,  it is clear that $S_1^{\rm acc}(\rho_A) \le S_1(\rho_A)$. 
Moreover, the difference 
%
\begin{equation}
    \Delta S_1 (\rho_A) \equiv S_1(\rho_A) - S_1^{\rm acc}(\rho_A)
    \label{eq:DeltaS1}
\end{equation}
%
can be determined  by noting that the superselection rule guarantees that $[\rho_A,\hat{n}]=0$ where $\hat{n}$ is the number operator acting in partition $A$. Thus $\rho_A$ is block-diagonal in $n$ and it can be shown \cite{Klich:2008se} that 
%
\begin{equation}
    \Delta S_1 (\rho_A) =  H_1(\{P_n\})
\label{eq:DS1H1}
\end{equation}
%
where
%
\begin{equation}
    H_1(\{P_n\}) = -\sum_{n=0}^N P_n \ln P_n.
\label{eq:H1}
\end{equation}
%
is the Shannon entropy of the number probability distribution.  It is instructive to consider \Eqref{eq:DS1H1} for the special case of a discrete Gaussian distribution, $P_n \propto e^{-(n-\expval{n})^2/2\sigma^2}$ where $H_1 = \ln\left(2\pi e \sigma^2 + \tfrac{1}{12}\right)$ depends only on the variance of $P_n$
%
\begin{equation}
    \sigma^2 \equiv \expval{n^2}-\expval{n}^2 = \sum_{n=0}^N n^2 P_n  - \left(\sum_{n=0}^N n P_n\right)^2\,.
\label{eq:sigma2}
\end{equation}
%
Thus, when the number fluctuations are Gaussian, the von Neumann accessible entanglement is completely determined by the variance.

% ---------------------------------------------------------------------------
\subsection{\ren Accessible Entanglement: $\alpha \ne 1$}
% ---------------------------------------------------------------------------

Computing the accessible entanglement for a many-body system is a difficult task for $\alpha=1$, as full state tomography is required to reconstruct the density matrix $\rho$. However, for integer values with $\alpha > 1$ a replica trick can be used to recast $\mathrm{Tr} \rho_A^\alpha$ as the expectation value of some local operator \cite{Calabrese:2004hl}. This advance has led to a boon of new entanglement results using both computational \cite{Hastings:2010dc, Humeniuk:2012cq, McMinis:2013dp, Herdman:2014ey, Drut:2015fs} and experimental \cite{Daley:2012bd, Islam:2015cm, Kaufman:2016ep, Pichler:2016ec, Linke:2017tf, Lukin:2018wg} methods.  Motivated by this progress, two of us recently generalized the accessible entanglement to the case of \ren entropies with $\alpha \ne 1$ and found that \cite{Barghathi:2018oe}:
%
\begin{equation}
S_{\alpha}^{{\rm acc}} (\rho_A) = \frac{\alpha}{1-\alpha}\ln\left[ \sum_{n} P_n \mathrm{e}^{\frac{1-\alpha}{\alpha} S_{\alpha}(\rho_{A_n})}\right]
\label{eq:S_alpha_acc}
\end{equation} 
%
% which reproduces Eq.~\eqref{eq:S1acc} in the limit $\alpha \to 1$. While not physically transparent in this form, the modification from the $\alpha=1$ case results from replacing the geometric mean in Eq.~\eqref{eq:S1acc} with a general power mean whose form is constrained by the physical requirement that $0 \le \Delta S_\alpha \le \ln (N+1)$.  It can also be interpreted as the quantum generalization of the conditional classical \ren entropy \cite{Cachin97entropymeasures,GolshaniPashaYari:2009,Hayashi:2011,SKORIC:2011el,FehrBerens2014} subject to physical constraints \cite{Barghathi:2018oe}. The arguments leading to Eq.~\eqref{eq:DS1H1} can also be generalized leading to 
which reproduces Eq.~\eqref{eq:S1acc} in the limit $\alpha \to 1$. While not physically transparent in this form, the modification from the $\alpha=1$ case results from replacing the geometric mean in Eq.~\eqref{eq:S1acc} with a general power mean whose form is constrained by the physical requirement that
%
\begin{equation}
 0 \le \Delta S_\alpha \le \ln (N+1)
\label{eq:DeltaS_alpha_inq}
\end{equation}
%
where the upper bound is equal to the support of $P_n$. \Eqref{eq:S_alpha_acc} can also be interpreted as the quantum generalization of the conditional classical \ren entropy \cite{Cachin97entropymeasures,GolshaniPashaYari:2009,Hayashi:2011,SKORIC:2011el,FehrBerens2014}, subject to physical constraints \cite{Barghathi:2018oe}. The arguments leading to Eq.~\eqref{eq:DS1H1} can then be generalized (see the supplemental material of Ref.~[\onlinecite{Barghathi:2018oe}]) leading to 
%
\begin{equation}
    \Delta S_{\alpha}\equiv  S_{\alpha} - S_{\alpha}^{{\rm acc}} = H_{{1}/{\alpha}}\left(\{P_{n,\alpha}\}\right)
\label{eq:S_alpha_acc5}
\end{equation}
%
where we introduce the classical \ren entropy of $P_n$
%
\begin{equation}
    H_{\alpha}\left(\{P_n\}\right)=\frac{1}{1-\alpha}\ln\sum_n P_n^{\alpha} 
\label{eq:Halpha}
\end{equation}
%
and
%
\begin{equation}
    P_{n,\alpha}=\frac{\Tr\, \left[{\mathcal{P}}_{A_{n}} \rho_A^{\alpha} {\mathcal{P}}_{A_{n}}\right]}{\Tr\, \rho_A^{\alpha}}=\frac{ P_n^\alpha\Tr\, \rho_{A_n}^{\alpha}}{\Tr\, \rho_A^{\alpha}}
\label{eq:Pna}
\end{equation}
%
can be interpreted as a normalization of partial traces of $\rho_A^{\alpha}$, where the SSR fixing the total particle number leads to $\Tr\, \rho_A^{\alpha}=\sum_n \Tr\, \left[{\mathcal{P}}_{A_{n}} \rho_A^{\alpha} {\mathcal{P}}_{A_{n}}\right]$ and thus guarantees the normalization of $P_{n,\alpha}$. Note that we have defined $P_{n,1} \equiv P_n$ for notational consistency. For brevity, let $H_{\alpha}(\{P_n\}) \equiv H_{\alpha}$ from here onwards. 

Writing the difference $\Delta S_{\alpha}$ as the classical \ren entropy of the fictitious probability distribution $P_{n,\alpha}$, simplifies the calculation of $\Delta S_{\alpha}$ and clarifies its properties, e.g., the fact that $H_{\alpha}$  is positive and bounded from above by $ H_{0}=\ln(N+1)$ guarantees that $\Delta S_{\alpha}$ satisfies the physical requirement in \Eqref{eq:DeltaS_alpha_inq}. \cite{Barghathi:2018oe} In addition, $P_{n,\alpha}$ is fully determined by $P_n$ and the full and the projected traces of $\rho_A^{\alpha}$, i.e.~$\Tr\, \rho_{A}^{\alpha}$ and $\Tr\, \rho_{A_n}^{\alpha}$, which can be measured using the experimental and numerical methods mentioned above.    